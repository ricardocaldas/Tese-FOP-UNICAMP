%% Use \documentclass[print]{dissertation} to export the document
\documentclass[]{tesefop}



%%%%%%%%%%%%%%%%%%%%%%%%% Configuração: dados pessoais %%%%%%%%%%%%%%%%%%%%%%%%%

% FIXME Substituir 'Nome completo do aluno' pelo seu nome.
\newcommand{\autor}{Nome completo do aluno}
% FIXME Se for do sexo feminino, descomente a linha a seguir.
 %\def\femaleAuthor{}

% FIXME Substituir 'Título da defesa' pelo título da defesa.
\newcommand{\titulo}{Título da defesa}

% FIXME Substituir 'Título do Doutorado".
\newcommand{\titulodoc}{Odontologia}

% FIXME Substituir 'Area do doutorado'
\newcommand{\areadoc}{Dentes}


% FIXME Substituir 'Nome completo do orientador' pelo nome completo do seu
% orientador.
\newcommand{\orientador}{Nome completo do orientador}
% FIXME Se for orientado por uma mulher, descomente a linha a seguir.
% \def\femaleOrientador{}

% FIXME Substituir 'Nome completo do coorientador' pelo nome completo do seu
% coorientador. Caso não tenha coorientador, comente a linha a seguir.
\newcommand{\coorientador}{Nome completo do coorientador}
% FIXME Se for coorientado por uma mulher, descomente a linha a seguir.
% \def\femaleCoorientador{}

% FIXME Substituir 'Ano' pelo ano em que ocorreu sua defesa.
\newcommand{\ano}{Ano}

  % Dados pessoais e da tese



\begin{document}
	
	%elementos pré-textuais. Não há necessidade de modificar este arquivo.
	\thispagestyle{plain}
\noindent% just to prevent indentation narrowing the line width for this line
\includegraphics[width=0.10\textwidth]{imagens/logo-unicamp}%
\begin{minipage}[b]{0.7\textwidth}
	\centering
	\textbf{UNIVERSIDADE ESTADUAL DE CAMPINAS} \\
\vspace{0.5cm}

\textbf{FACULDADE DE ODONTOLOGIA DE PIRACICABA}
\end{minipage}%
\includegraphics[width=0.10\textwidth]{imagens/logo-fop}

\vspace{4cm}
\begin{center}
	% O tamanho da fonte deve ser 16pt.
	% Deve-se utilizar caixa alta.
	{\Large\textsc{\autor}}
\end{center}
\vspace{4cm}
\begin{center}
	% O tamanho da fonte deve ser 16pt em negrito.
	% Deve-se utilizar caixa alta.
	{\Large\textbf{\textsc{\titulo}}}
\end{center}
\vfill
\begin{center}
	% O tamanho da fonte deve ser 12pt em negrito.
	% Deve-se utilizar caixa alta.
	\textbf{Piracicaba \\ \ano}
\end{center}





\cleardoublepage
% Folha de rosto


\thispagestyle{plain}


\begin{center}
	
	{\large\textbf{\textsc{\autor}}}
	
	
\end{center}
\vfill


\vfill
\begin{center}
	{\Large\textbf{\textsc{\titulo}}}
\end{center}
\vfill

\begin{flushright}
	\begin{minipage}[c]{.5\textwidth}
		\ifx\mestrado\undefined
		Tese apresentada à Faculdade de Odontologia de Piracicaba da Universidade Estadual de Campinas como parte dos requisitos para obtenção do título de \ifx\femaleAuthor\undefined
		Doutor
		\else
		Doutora
		\fi
		em \titulodoc{} na área de \areadoc . 
		\else
		Dissertação apresentada à Faculdade de Odontologia de Piracicaba da Universidade Estadual de Campinas como parte dos requisitos para obtenção do título de \ifx\femaleAuthor\undefined
		Mestre
		\else
		Mestra
		\fi
		em \titulodoc{} na área de \areadoc .
		\fi

		
		\end{minipage}
\end{flushright}
\vspace{.5cm}

\noindent
\textbf{Orientador\ifx\femaleOrientador\undefined
	\else
	a\fi: \orientador
}
\vspace{.25cm}

\ifx\coorientador\undefined
\else
\noindent
\textbf{Coorientador\ifx\femaleCoorientador\undefined
	\else
	a\fi: \coorientador
}
\vspace{.5cm}
\fi

\noindent
\begin{minipage}[c]{.5\textwidth}
	{\footnotesize\textsc{Este exemplar corresponde à versão final da
			\ifx\mestrado\undefined
			tese
			\else
			dissertação
			\fi
			defendida
			\ifx\femaleAuthor\undefined
			pelo aluno
			\else
			pela aluna
			\fi
			\autor,
			e orientada pel\ifx\femaleOrientador\undefined
			o\else
			a\fi{} Prof\ifx\femaleOrientador\undefined
			\else
			a\fi. Dr\ifx\femaleOrientador\undefined
			\else
			a\fi. \orientador.
		}}
	\end{minipage}
	\vspace{1cm}
	
	\noindent
	
	\vspace{.5cm}
	
	
	\vfill
	\begin{center}
		{\small\textbf{\textsc{ Piracicaba \\ \ano}}}
	\end{center}


\clearpage


\includepdf{ficha-catalografica}

\includepdf{folha-de-aprovacao}

	
	
	\section{Dedicatoria}



\blindtext %apague completamente esta linha.
 % Opcional. Remova esta linha caso não queira incluir
	\section{Agradecimentos}

\blindtext %apague completamente esta linha. % Opcional. Remova esta linha caso não queira incluir
		
	\section{Resumo}

\Blindtext %apague completamente esta linha.
	\section{Abstract}
\noindent % Parágrafo sem  indentação
This study assessed the reliability of multidetector computed
tomography (MDCT) in determining the surgical risk to the inferior
alveolar nerve (IAN) in extractions of third molars. Our sample comprised  ... \\ %deixar as duas barras para aumentar o espaço entre o fim do texto e as palavras-chave 

\noindent \textbf{Key-words:}
word 1, word 2, word 3
	\section{Outro abstract}

\Blindtext %apague completamente esta linha. % Remova esta linha caso não inclua o resumo extra 
	
	\section{Lista de Ilustrações}

\blinditemize
 % Opcional. Remova esta linha caso não queira incluir
	\section{Lista de Tabelas}

\blinditemize % Opcional. Remova esta linha caso não queira incluir
	\section{Lista de Abreviaturas e Siglas}

\blinditemize % Opcional. Remova esta linha caso não queira incluir
	\section{Lista de Símbolos} % Opcional. Remova esta linha caso não queira incluir
	
	\include{sumario}

	\include{introducao}
	
	%%%%%%%%%%% Formato Tradicional %%%%%%%%%%%%%%%%
	% Instruções: Caso opte pelo formato tradicional, remova o % no inicio de cada linha a seguir
	%\include{revisao-literatura}		% Formato tradicional, remova % no inicio desta linha.
	%\include{proposicao}				% Formato tradicional, remova % no inicio desta linha.
	%\include{materiais-metodos}		% Formato tradicional, remova % no inicio desta linha.
	%\include{resultados}				 % Formato tradicional, remova % no inicio desta linha.
	
	%%%%%%%%%%% Formato Alternativo  %%%%%%%%%%%%%%%%
	
	% Instruções: Caso opte pelo formato alternativo, remova o % no inicio de cada linha a seguir
	%\include{artigos/artigo1}		% Formato alternativo, remova % no inicio desta linha.
	%\include{artigos/artigo2}		% Formato alternativo, remova % no inicio desta linha.
	%\include{artigos/artigo3}		% Formato alternativo, remova % no inicio desta linha.

	
	
	
	\include{discussao}
	\include{conclusao}
	
	\include{referencias}
	\include{apendice1}
	
	\include{anexo1}
	
				
	
	

\end{document}